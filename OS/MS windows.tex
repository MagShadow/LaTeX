% homework
% 20161005
% by hiaoxui

\documentclass[a4paper,12pt,notitlepage]{article}

\usepackage{CJKutf8}
\usepackage{indentfirst}

\setlength{\parindent}{2em} 

\begin{CJK*}{UTF8}{gbsn}
\begin{document}

\title{Microsoft Windows演变过程}
\author{整理:hiaoxui}
\maketitle

\section{起源}

	1970年,美国施乐公司成立了著名的研究机构帕罗奥多研究中心,施乐于 1981 年宣布推出世界上第一个商用的图形用户界面系统“Star 8010”工作站.但由于种种原因,此技术上并未得到大众的重视,也没有协助商业化的应用. \\
	
	这时苹果电脑的创始人之一的史蒂夫·乔布斯在参观施乐公司的帕罗奥多研究中心后认识到图形用户界面的重要性以及广阔的市场前景,便开始着手进行自己的图形用户界面系统.苹果公司先后推出了Lisa OS以及System Software系统,成为了世界上第一个成功的商用图形用户界面系统. \\
	
	苹果电脑在开发 Macintosh 时基于市场战略上的考虑,故意开发了只能在苹果电脑自己的电脑上作运作的图形用户界面系统.微软希望能够创建业界标准,希望它能够成为基于intel x86未处理新编计算机上的标准用户界面操作系统.于是微软开始着手开发自己的有图形界面的操作系统,MS Windows. \\

\section{Windows1.x}

	Windows1.0在1985年11月20日发售,这是Windows的第一款产品,也是微软的第一款图形界面操作系统.但是Windows1.0其实不是完整的操作系统,只是基于MS-DOS的拓展.它需要两张软盘和192KB的内存,微软把它描述为MS-DOS2.0的一个驱动程序.Windows 1.0中鼠标作用得到特别的重视,用户可以通过点击鼠标完成大部分的操作.Windows 1.0自带了一些简单的应用程序,包括日历、记事本、计算器等等.总之,看那时的Windows 1.0,总会让人感到它像是一个PDA,不过这在当时已经相当吸引人了.Windows1.0是支持多任务执行的,这在当时看来非常有吸引力. \\
	
	Windows一直着眼于它的交互模式,这有利于其程序的稳定性.由于微软支持软件的向后兼容,它不仅可以执行Windows当前版本上的二进制程序,还能执行之前版本的二进制程序,只需要做微小的修改即可.它被视作"front-end to the MS-DOS operating system",可以直接调用MS-DOS的功能,并且GUI程序可以从.exe执行.但是,Windows1.0有自己的新的可执行文件"NE",仅适用于Windows. \\
	
	从操作系统的发展角度来看,Windows已经实现了一个基于软件的虚拟内存机制,这样软件就可以使用比RAM大的内存来运行.这在当时内存在KB量级的计算机界是很大的一个进步,Windows也因此而被开发出了更大的潜力,因而可以支持多任务和GUI界面. \\
	
	从计算机发展角度来看,Windows已经使用了一直沿用至今的大规模集成电路.这大大缩小了其体积,为其成为世界上最流行的个人计算机奠定了基础. \\
	
\section{Windows2.0}

	1987年微软发行Windows 2.0,比起上一版本较受欢迎.主要原因是微软发行“运行时期版本”的Excel 和 Word for Windows,即是程序可于MS-DOS运行,然后自动引导Windows,结束程序时同时关闭Windows. \\
	
	2.0x 版本使用实模式记忆模式,限制了最多只可运用1M内存.后期再发行两个版本,分别为Windows/286 2.1 和 Windows/386 2.1.Windows/286 2.1 仍使用实模式记忆模式,但首次支持HMA.Windows/386则使用保护模式记忆模式,加上EMS模拟. \\
	
	Windows2.0版本依然只有十六位色的GUI界面,但是相比Windows1.0,Windows2.0支持窗口的重叠,还支持组合按键和窗口的缩放等.正如Windows1.0一样,Windows2.0的软件在没有修改的情况下无法在更高版本的Windows上运行. \\
	
	从计算机发展的角度来看,Windows2.0首先支持了VGA显卡,虽然依然只有16位色,但是这大大增强了其图形性能. \\
	
	从操作系统发展的角度来看,囿于当时还没有成熟的硬盘设计技术,Windows2.0并不支持硬盘,但是它是第一个支持EMS内存的操作系统.Windows2.0是为Intel 286设计的,但是在Intel 386发布之后,Windows2.0迅速利用了新的处理器拓展内存的优势,Windows2.0的速度和可靠性得到了巨大提升. \\
	
\section{Windows3.0}

	微软的Windows系列操作平台的Windows 3.x家族发行于1990年到1994年间,其中的3.0版是第一个在世界上获得成功的版本,使得微软的操作系统可以和苹果电脑公司的麦金塔电脑以及在图形化界面的Commodore的Amiga竞争.Windows 3.x基于MS-DOS操作系统. \\
	
	Windows 3.0相比Windows 2.0而言,有显著提升.Windows的图标和图形支持全部16种颜色的EGA和VGA模式,而Windows 2.x的只有有色菜单和窗口盒,且颜色非常有限.256色VGA模式被首次被支持. \\
	
	从操作系统的角度出发,Windows 3.x依然是基于MS-DOS的操作系统.它在3.1版本时添加了对声音输入输出的基本多媒体的支持和一个CD音频播放器,以及对桌面出版很有用的TrueType字体. \\
	
	Windows 3.x还拥有简单的网络功能,虽然它的TCP/IP依靠于第三方软件,例如Trumpet Winsock.微软还对Windows 3.x开发了一个叫Win32s的附加组件,对Win32 API提供有限的支持. \\
	
	Windows 3.x具备了模拟32位操作系统的功能,图片显示效果大有长进,对当时最先进的386处理器有良好的支持.这个系统还提供了对虚拟设备驱动(VxDs)的支持,极大改善了系统的可扩展性,计算机用户再不必在购买Windows 3.x时煞费苦心地查证自己的硬件是否可以被系统支持了,因为它完全可以另外安装一个驱动程序. \\
	
	Windows 3.2开始可以播放音频、视频,甚至有了屏幕保护程序和传真.值得一提的是,从Windows 3.x版本开始,Windows开始关注中国市场,并推出 了中国版本的Windows 3.1.这为Windows 95在国内的成功打下了基础. \\
	
	从计算机发展角度来看,Windows 3.x版本不再要求用户提供两块软盘,而可以在硬盘运行.这背后是计算机存储技术的飞速发展. \\
	
\section{Windows 95}

	Windows 95中(代号芝加哥)是由微软开发的面向消费者的操作系统.它被发布于1995年8月24日,相比公司先前基于DOS的Windows产品有显著的改善. \\

	Windows 95的合并微软以前独立的MS-DOS和Windows的产品.它相比它的前身Windows 3.1,最主要的是在图形用户界面(GUI)有很大改观,并在其简化的“插件和播放”功能特色有显著改善. \\

	Windows 95 在操作系统的设计上和前代产品有很大不同,如从主要合作式多任务的16位架构迁移到先发制人多任务的32位架构(windows 95的代码是16位和32位混合的),操作系统的核心组件取得了重大的变化.之所以依然保留16位代码,主要是为了和MS-DOS兼容 \\
	
	Windows 95的架构是Windows的工作组386增强模式的演变.操作系统的底层由大量虚拟设备组成,它们运行在32位保护模式和虚拟8086模式.所述虚拟设备驱动程序负责处理物理设备(诸如视频和网络卡),模拟由虚拟机或提供各种系统服务所使用的虚拟设备.这些设备中最重要的是: 
	
\begin{enumerate}
	\item VMM32.VXD
	\item CONFIGMG
	\item Input/Output Subsystem
\end{enumerate}

	Windows 95的系统已经采用了一些更为先进存储保护技术,32位的Windows程序被分配其自己的存储器段,它可以调节到任何所需的大小.外段存储器区域不能由程序进行访问.这样如果它崩溃了的话,就不会破坏系统的稳定性. \\

	Windows 95在处理器保护上也有很大进步.Windows 95充分利用386处理器的能力,支援两个特权级.它用0和3两个特权级管理微处理器,也可称为两个环.环0中的部件是操作系统的底层,如包括对低级内存储器管理的支持,环0里的软件在整个系统中功能最强,包括了几乎所有微处理器的指令,并能存取关键的数据结构,如页表等.因此环0里的软件最可靠. \\

	Win32 API是Microsoft公司的战略性系统接口,它第一次出现在Windows NT中,并把其子集Win32 API引入到Windows 3.1中.正是由于Win32 API的强大功能及远大前途,Windows 95也包含了Win32.Win 32的API由三个模块组成,每个模块包括一个16位和32位的组件. \\
	
\begin{enumerate}
	\item Kernel:提供内存,进程和文件系统的通道
	\item User:管理用户界面
	\item GDI:绘制图形
\end{enumerate}

	Windows 95 的系统还有一个更大的进步,就是脱离了MS-DOS系统.对于终端用户而言,MS-DOS还是windows系统的基础.比如它可以阻止GUI程序的加载,也可以使系统进入纯DOS环境.但是当GUI界面打开之后,虚拟机管理系统就接替了DOS的功能,此时DOS只被用作16位程序的兼容系统.这与早期版本的Windows依赖于MS-DOS执行文件和磁盘访问形成对比.这对Windows系统而言是一个里程碑式的变化. \\
	
	由于计算机硬件,特别是网络技术的飞速发展,Windows95也拥有了更新的联网技术,提供简捷的网络浏览方式、资源共享方式、网络安装和配置操作、拨号网络和远程网络管理等.拨号网络功能利用调制解调器,实现网络互联.通过远程管理功能,网络系统管理员可以查看正在使用的计算机上的文件,并帮助解决问题.丰富多彩的Windows设计方案. \\

\section{Windows98}

	Windows 98于1998年五月十五日发售,像它的前身时,Windows 98是一个混合的16位和32位单片式产品,基于MS-DOS的引导程序. \\
	
	Windows 98是第一个使用Windows驱动程序模型(WDM)的操作系统.但是它并没有做很好的宣传,导致很多厂商依然在做旧的VxD标准的驱动.WDM在之后的Windows 2000 和 Windows xp中得到了广泛应用. \\
	
	从操作系统技术进步的角度来讲,Windows 98有很多新的特性值得注意.随着互联网的诞生,Windows 98引入了Web-aware技术和Suite of Tools for Internet Communication,使得操作系统和因特网融为一体.众所周知,网络功能和操作系统的耦合程度是评价一个操作系统优劣的重要指标.所以Windows 98在这方面是一个极其成功的范例. \\
	
	Windows 98还首先支持了升级管理机制和VPN,这些都是之前的Windows操作系统所不支持的.此外它还改进了文件系统,从FAT过渡到了FAT32文件系统,使用户可以获得更多的空间. \\
	
	Windows 98的推出ACPI1.0支持,这使待机(ACPI S3)和Hibernate(ACPI S4)状态成为可能.然而,休眠支持是非常有限的,取决于具体的供应商.睡眠仅当兼容(PNP)的硬件和BIOS都存在,硬件制造商或OEM提供兼容的WDM驱动程序时才能使用. \\
	
	从计算机硬件进步的角度来看,Windows 98仍然是一个成功的操作系统.随着计算机接口的标准化和USB技术的进步,Windows 98中有比Windows95只曾在OEM版本的支持(OSR 2.1或更高版本)更强大的USB支持(例如,用于USB复合设备支持).Windows 98 支持USB集线器,USB扫描仪和影像类的设备.Windows 98中还引入了内置的一些USB人机接口设备类(USB HID)和PID类设备,如USB鼠标,键盘支持下,通过一定数量的消费者页的力回馈摇杆等,包括额外的键盘功能的HID控制. \\
	
	随着计算机硬盘的容量飞速增长,FAT格式的硬盘已经不再满足人们的需求.所以Windows 98开始支持FAT32文件系统.这种文件系统可以格式化2G以上的硬盘. \\
	
	而伴随着互联网飞速进步的是Windows 网络体验的迅速增强.Windows 98取得成功的重要因素就是其网络支持非常好. \\
	
	Windows 95的推出了32位保护模式缓存驱动程序, VCACHE 代替  SMARTDRV 来缓存在内存中的硬盘驱动器最近访问过的信息,分成块.但是,高速缓存参数需要手动调整,因为它通过占用过多的内存,而释放它的速度不够快,使得系统出现性能下降..在FAT32文件系统,Windows98有一个名为 MapCache 的功能,可以从磁盘缓存本身如果可执行文件的代码页对齐/映射到4K的边界,而不是将它们复制到虚拟内存运行的应用程序.这将导致更多的可用的内存运行的应用以及交换文件的较少的使用. \\
	
\section{Windows ME}

	Windows ME在2000年6月发售.这个系统是在Windows 95和Windows 98的基础上开发的,在内核方面没有实质性的进展,但它包括一些相关的小改善.尤其是用户界面方面为之后的Windows XP提供了一些借鉴. \\
	
	系统不再包括实模式的MS-DOS:这就意味着,与Windows 95和98不同,微软在加载Windows图形界面前隐藏了加载DOS的过程,使得启动时间有所减少.它仍然提供DOS模式,可以运行在窗口中,但是一些应用程序(如较早的磁盘工具)需要实模式,而不能运行在DOS-窗口中.微软摒弃了Windows Me的DOS实模式,理论上这有助于系统的速度提升,减少了对系统资源的使用.然而事实上这反而对基于DOS的Windows Me造成了不利影响,系统不稳定,而且造成了Windows Me运行非常慢.以至于在使用Windows Me一段期间后,系统就有明显变慢,甚至不停出现蓝白画面甚至死机等异常现象. \\
	
	Windows ME 提供了系统还原日志和还原系统.这意味着简化了故障排查和问题解决工作.理论上,这是一个大的改进:用户不再需要难学难懂的DOS行命令的知识就可以维护和修复系统.可实际上,失去DOS功能对维护来说是一个障碍,而系统还原功能也带来一些新麻烦:性能显著的降低;实际证明系统还原并不能有效胜任一些常见的错误修复任务,例如程序会在系统还原后无法运行、系统死机的情况也更严重.由于系统每次都自动创建一个先前系统状态的备份,使得非专业人员很难完成一些急需的修改,甚至包括删除一个不想要的程序或病毒,所以很多杀毒软件在消除病毒及一般使用下不建议打开该系统还原功能. \\
	
\section{Windows 2000}

	Windows 2000(简称Win2K)是一个独占式、可中断运行、具有图形用户界面(GUI)和商业导向的操作系统.Windows 2000可以运作在单处理器系统或者多处理系统(SMP)上,是属于微软Windows NT产品线的一部分,发布于2000年2月17日,此外Windows 2000也是Windows XP(2001年10月上市)和Windows Server 2003(2003年4月上市)的前身. \\
	
	从操作系统的进步角度来讲,Windows 2000的内核采用混合式核心设计,Windows NT的产品线也都使用这种核心模式.相较于之前仅管理OSI第二层的模式,Windows 2000的网络管理方面可以管理到OSI第三层. \\
	
	Windows 2000是高度模块化的系统,系统中包含了两个层次:用户模式和核心模式.用户模式顾名思义代表了用户程序在这个模式中运行,只能访问部分的系统资源.核心模式则可以访问所有的系统资源,包含了内存和外部的设备.用户模式运行程序是通过"Executive"的接口来运行,而Executive本身是位于核心模式,因此可以访问到各种系统资源. \\
	
	Windows 2000中还引入了一个分布式链接跟踪服务,以确保文件的快捷方式保留即使目标移动或重命名工作.目标对象的唯一标识符存储在NTFS3.0的快捷方式文件中,Windows可以使用分布式链接跟踪服务跟踪快捷方式的目标,这样,如果目标移动,甚至到另一个硬盘驱动器,该快捷方式的文件会被更新. \\
	
	从计算机硬件技术进步的角度来讲,用户需要更大的存储空间.Windows支持NTFS 3.0格式的硬盘,NTFS格式的硬盘支持4G以上的独立文件,而FAT32只能支持4G以下大小的文件. \\
	
\section{Windows XP}

	Windows XP(开发代号:Whistler)是微软公司推出供个人电脑使用的操作系统,包括商用及家用的桌上型电脑、笔记本电脑、媒体中心和平板电脑等.其RTM版于2001年8月24日发布;零售版于2001年10月25日上市.其名字XP的意思是英文中的体验(experience).Windows XP是继Windows 2000及Windows Me之后的下一代Windows操作系统,也是微软首个面向消费者且使用Windows NT架构的操作系统. \\
	
	Windows XP有很多新功能,包括但不限于:
	
\begin{itemize}
	\item 更快的引导与休眠过程
	\item 提供驱动程序恢复功能以应对由于更新或升级设备驱动程序可能造成的问题
	\item 提供更加友好的用户界面,以及为桌面环境开发主题的架构
	\item 快速切换用户,允许一个用户存储当前状态及已打开的程序,同时允许另一用户在不影响该等信息的情况下登录
	\item ClearType字体渲染机制,用以提高液晶显示器上的文字可读性
	\item 远程桌面功能允许用户通过网络远程连接一台运行Windows XP的机器操作应用程序、文件、打印机和设备
	\item 支持多数DSL调制解调器以及无线网络连接,以及通过火线和蓝牙的网络连接
\end{itemize}

	从计算机硬件发展的角度来讲,Windows XP开始发行64位版本的操作系统.随着内存的扩大,越来越多的用户选择使用大于4G容量的内存,然而32位版本的Windows XP系统只能利用到其中4G的容量,所以64位版本的Windows XP是一个极具标记性意义的产品. \\
	
	2011年9月底前,Windows XP是世界上使用人数最多的操作系统,市场占有率达42\%;在2007年1月,Windows XP的市场占有率达历史最高水平,超过76\%.根据Netmarketshare公司对全球互联网用户的统计数据显示,2012年8月份,统治操作系统市场长达11年之久的Windows XP最终被Windows 7超越. \\

	Windows XP成为有史以来销量最大及最常用的操作系统,包括历史占有率最高,超高的人气更令Windows XP的下一代,于2007年1月30日上市的Windows Vista惨淡收场. \\
	
\section{Windows 7}

	Windows 7于2009年7月22日发放给组装机生产商(OEM),零售版于2009年10月23日在中国大陆及台湾发布,香港于翌日发布. \\
	
	Windows 7提高了屏幕触控支持和手写识别,支持虚拟硬盘(VHD),改善多核心(Multi-Core)处理器的运作效率,开机速度和内核改进.增加的功能大致上包括:支持多个显卡、新版本的Windows Media Center、一个供Windows Media Center(WMC)使用的桌面小工具、增强的音频功能、自带的XPS和Windows PowerShell以及一个包含了新模式且支持单位转换的新版计算器.另外,其控制台也增加了不少新项目:ClearType文字调整工具、显示器色彩校正向导、桌面小工具(Desktop Gadget)、系统还原(System Restore)、疑难解答、工作空间中心(Workspaces Center)、认证管理员、系统图标和显示.旧有的Windows安全中心被更名为“Windows行动作业中心”,它有保护电脑信息安全的功能. \\
	
	对开发者来说,Windows 7提供一套全新的网络API.这些API支持使用机器语言创建基于SOAP的网络服务(而非基于.NET的WCF网络服务).此外,新的操作系统缩短应用程序安装所需的时间,对UAC进行改进,用户可以自己调节UAC,以减少UAC(User Account Control,用户帐户控制)提示的出现次数,简化安装时的安装过程,并对API增加不同语言的支持. \\
	
	在Windows 7 Professional、Enterprise和Ultimate三个版本中会含有一个类似虚拟微机“Windows Virtual PC”的功能.这种虚拟微机能让用户在Windows 7运行的同时运行不同的Windows环境,包括“Windows XP模式”.Windows XP模式可让用户在一台虚拟机上运行Windows XP,并将其正在运行的程序显示于Windows 7的桌面上. Windows 7允许用户安装一个虚拟硬盘作为一般的数据储存介质,并可利用Windows 7自带的引导装置从虚拟硬盘中读取并运行Windows系统.Windows 7的远程桌面控制功能也有改善,它支持运行3D游戏、视频播放等多媒体程序,同时DirectX 10也可以在远程桌面(Remote Desktop) 环境使用.另外,原本的Windows Vista Starter限制用户只能同时运行三个程序,但在Windows 7 Starter中,这个限制已经取消. \\
	
	Windows 7涵盖32位和64位二个版本,顾及从32位系统过渡到64位系统的趋势.Windows Server 2008 R2则只对应64位服务器系统,但亦会兼容32位程序.而16位窗口系统和MS-DOS应用程序,则提供有限度支持,情况如同Windows Vista x64版本. \\
	
	从计算机硬件进步的角度来讲,Windows 7对CPU有极高的依赖.在早期版本中甚至必须要求CPU支持虚化,但此项要求已经于2010年3月20日由微软所发布的补丁解除. \\
	
\section{Windows 8.x}

	Windows 8是微软公司于2012年推出电脑操作系统,采用与Windows Phone 8相同的NT内核,被认为是微软反击主导平板电脑及智能手机操作系统市场的苹果iOS和Google Android的操作系统. \\
	
	该操作系统除了具备微软适用于笔记本电脑和台式机平台的传统窗口系统显示方式外,还特别强化适用于触屏的平板电脑设计[,使用了新的接口风格Metro,新系统亦加入可通过官方网上商店Windows Store购买软件等诸多新特性. \\
	
	从计算机硬件进步的角度来说,Windows 8的出现迎合了触屏电脑的流行趋势.微软Windows 8因应触控潮流,使用接口作出革命性转变,而推出的Metro地铁风格接口,因为为了行动方面而做出的改变过快,导致桌面用户不习惯,加上虽然已有Metro接口存在,但是工作时却仍然要经常切换到桌面,在Windows RT上也是如此,这也是为什么Windows 8推出快一年,销售却仍然低迷的原因.而用户的困扰通常是没有开始按钮、没有闹钟、搜索功能不好用、Metro接口和桌面不搭、没有Windows Aero毛玻璃特效. \\
	
\section{Windows 10}

	Windows 10是一个由微软开发的操作系统,是Windows家族的成员,为Windows 8.1和Windows Phone 8.1的后继者,开发代号为Threshold和Redstone,Windows 10的设计目标是统一包括个人电脑、平板电脑、智能手机、嵌入式系统、Xbox One及Surface和Microsoft HoloLens等等,整个Windows产品系列的操作系统.它们共享一个通用的应用程序架构和Windows Store的生态系统. \\
	
	从计算机硬件进步的角度来讲,Windows 10的出现意味着VR,触屏电脑和游戏主机等产品的普及和逐渐成熟.而微软也在着手打造它自己的生态圈和产品链.它覆盖所有尺寸和品类的Windows设备,可无缝运行于微型计算机(类似英特尔伽利略)、手机(ARM芯片)、平板(ARM和x86芯片)、二合一设备、桌面电脑及服务器等几乎所有硬件.Windows 10贯彻了“移动为先,云为先”(mobile first, cloud first)的设计思路,三屏一云,多个平台共用一个应用商店,应用统一更新和购买,是跨平台最广的操作系统. \\
	
	从操作系统进步的角度来讲,Windows 10重视不同设备之间的协调,尤其是在运行Windows版本的电脑和智能手机设备.Windows 10 电脑的Windows Runtime应用程序可以移植到Windows 10 Mobile,并分享代码库.Windows 10 Mobile与个人电脑版本分享同样的用户界面元素. \\
	
	2015年5月,在微软Ignite开发者大会上,该公司高层主管杰瑞·尼克逊透露,Windows 10将是这个Windows操作系统品牌的最后一版产品.Windows将不再有新版本问世,取而代之的将是定期的改进和更新.得以采用这种方式的部分原因是,把开始功能表和自带应用程序等操作系统部件分解为独立部分,各个部件可以独立更新,组成完整的Windows核心操作系统. \\
	
\end{CJK*}
\end{document}
