% homework
% 20161005
% by hiaoxui

\documentclass[a4paper,12pt,notitlepage]{article}

\usepackage{CJKutf8}
\usepackage{indentfirst}

\setlength{\parindent}{2em} 

\begin{CJK*}{UTF8}{gbsn}
\begin{document}

\title{OS X特征}
\author{整理:hiaoxui}
\maketitle

	我从没用过Mac OS,以下是我整理的一些关于OS X的资料. \\

	在Mac计算机中的一个重要组成部分,是固件.固件是编程的水平存在直接在硬件层的顶部.它不是操作系统本身的一部分.Mac的固件是是你打开Mac计算机上执行的第一个程序.它的工作是检查计算机的CPU,内存,磁盘驱动器和端口错误.Mac固件上的"PC"称为BIOS,它代表基本输入输出系统.假设没有通过固件报告的错误,Mac OS X第二个加载的程序是引导程序. \\
	
	在Mac OS X的核心是XNU内核.内核是指加载第一操作系统的一部分.它负责控制和监视硬件资源,如内存,CPU处理器分配和磁盘驱动器.该XNU内核包括从名为马赫旧计算机体系结构的系统代码.马赫是卡内基·梅隆大学的产品,出现于20世纪80年代.此代码负责Mac电脑中的一些基本功能,包括虚拟内存管理和多任务处理.该代码也会使得Mac OS的降低CPU的处理速度来避免硬件过热. \\
	
	内核的另一部分是输入输出(I/O)工具包.它依赖于C ++编程语言来控制设备驱动程序的专业化.设备驱动是让外部设备与计算机交互的程序.例如,你的打印机可能需要你的计算机上的设备驱动程序,让你可以从你的计算机打印.处理所有的请求和信息从电脑转移到其他设备是一个要求很高的工作.I / O Kit允许Mac计算机在同一时间用不同的技术处理多个设备.这就是为什么你可以连接使用USB,FireWire和Thunder电缆同时设备到Mac. \\
	
	在XNU内核的第三部分是计算机的安全警卫.这部分程序基于伯克利软件分发(BSD)衍生为UNIX,负责维护系统安全和权限.当你登录到Mac电脑时,BSD决定了你的访问级别.管理员有自由支配的权利并能下载或删除程序和其它数据.其他用户级别可能没有这样的自由权限.管理员可以定义什么内容一个普通用户不能访问.XNU内核的BSD元素也有助于Mac计算机的同步. \\
	
	还的Mac OS X的一部分是核心服务层和应用服务层.这些层形成了计算机工程师称之为堆栈的结构.堆栈只是概念化的计算机软件和硬件的各种层的关系的方式.在堆栈的底部,你会发现硬件和固件.下一级是操作系统内核.之后,你会发现核心服务,然后就是应用服务层.应用程序本身在堆栈的顶部. \\

	Mac OS X中的核心服务组件包含多个框架,允许计算机处理很多任务,包括在不同的语言的文本搜索任务等.应用服务层是基本图形用户界面(GUI)的系统.这个图形化环境是用户可以直接接触到的.应用服务层还充当各种应用程序之间的通信信道,允许它们之间相互调用. \\
	
	伴随着Mac OS X Lion的发行,苹果推出了一些新功能. Mac OS X支持多点触摸手势.这意味着如果你使用Mac电脑,无论是触控板或Magic Mouse鼠标,你可以使用多个触摸点,并执行特定运动执行某些命令.典型的例子是使用捏动放大缩小照片.在触摸板或魔术鼠标操作一定手势可以使得选择的图像缩小在屏幕上.Mac OS X系统预置了多个不同的手势. \\
	
	另一个特点是使用最初是为苹果运行iOS移动设备的全屏幕应用程序.苹果的产品如iPhone和iPad应用程序出现在了Store上.该公司现在允许Mac用户通过Mac App Store的在线购买应用程序. Mac OS X Lion可以显示这些应用程序全屏应用程序.你可以同时运行多个应用,并在它们之间切换. \\
	
	macOS通过提供一种称为Classic环境的模拟环境,保留了与较旧的Mac OS应用程序的兼容性,允许用户在macOS中把Mac OS 9当作一个程序进程来运行,使大部分旧的应用程序就像在旧的操作系统下运行一样.另外,给Mac OS 9和macOS的Carbon API可以创造出允许在两种系统运行的代码.OpenStep的API也依然可以使用,但是苹果现在把它称为Cocoa技术.(这个遗留下来的传统可以在Cocoa API中看到,大部分的类别名称都是以NeXTSTEP的缩写"NS"开头.)给开发者的第四个选项是可以在macOS当作"第一等公民"一样的Java平台上写应用程序—事实上这就是说Java应用程序尽可能的与操作系统合适地搭配而仍然能够"跨平台(cross-platform)",以及他的GUI,是以Swing撰写的,看起来几乎完全地与天生的Cocoa接口类似. \\
	
	只要他们能够在这个平台上被编译,macOS可以运行很多BSD或Linux软件包.编译过的代码通常是以macOS封装的方式来散布,但有些可能需要命令行的配置设置或是编译.像是Fink和DarwinPorts这样的项目,提供很多标准包之预先编译或是预先格式好的封装.在10.3版开始,macOS已经包含Apple X11,这是给Unix应用程序的X11图形接口的公司版本,当作是在安装阶段的选择性组件.苹果是以XFree86 4.3和X11R6.6为基础实现的,搭配一个模仿macOS外观的窗口管理器,与macOS有更密切的集成,延展扩充到使用天生的Quartz显像系统和加速OpenGL.早期的macOS版本可使用XDarwin来运行X11应用程序. \\
	
	想看看你的Mac上运行的所有应用程序?你可以使用任务控制来获取目前在计算机上运行的每个应用程序.这有点像Windows任务管理器的图形版本,你会看到表示为自己的窗口每个应用程序的列表.同样,快速启动功能,可以通过图标来表示每个应用网格布局.如果你在你打开快速启动时打开另一个应用程序,当你决定下一步要启动的应用程序是快速启动的窗口就会消失. \\
	
	Mac OS X Lion有一个特点,可能会保存你的皮肤.这里有一个自动保存功能,这有助于防止丢失工作.还有一个叫做version的功能,将显示你处理过的文档的历史功能.如果你你发现你正在执行的操作是错误的,你可以回去到一个较早的版本,并从那里开始,而不必扔掉整个文档. \\
	
	如果你想使用Mac电脑作为你的家庭网络服务器,OS X简化了安装过程.Mac有一个功能叫做AirDrop,你的Mac可以无线连接到任何其他使用AirDrop的电脑.你不需要开启一个共同的WiFi网络,这意味着你能够跨Mac电脑共享文件,而不必担心附近是否有网络. \\
	
	Mac OS X Lion中的VoiceOver功能让视障人士更方便地访问他们的计算机.例如,计算机可以大声读出文档.操作系统包括22种不同的语言的语音.还有一个冗长的盲文设置,让视障用户定义他们使用Mac上的各种应用程序时,到底需要多少返回信息. \\
	
	OS X还有若干功能,比如视频聊天的FaceTime,这些是苹果公司为iPhone设计的.OS X还有很多专有的APP.这些特性都是其他操作系统所不具备的. \\
	
\end{CJK*}
\end{document}
