% Physics experiment report
% 29/Oct/2016

\documentclass[a4paper,10pt,notitlepage]{article}

\usepackage{CJKutf8}
\usepackage{amsmath}
\usepackage{indentfirst}
\usepackage{graphicx}

\setlength{\parindent}{2em} 

\begin{CJK*}{UTF8}{gbsn}
\begin{document}

\title{观察光的偏振现象}
\author{秦光辉\ 9组3号}
\maketitle

\section{实验现象}

\subsection{用偏振光镜验证布儒斯特定律}

\subsubsection{先使得A//P,绕z轴转动A,观察A的反射光强度变化}

	随着A的转动,反射光强逐渐减弱,到90$^{\circ}$时消光,之后继续增强,在180$^{\circ}$时达到最大,然后继续减弱,到270$^{\circ}$时消光,之后继续增强,在0$^{\circ}$时回到原点,光强达到最大. \\
	
	激光以布儒斯特角入射到玻璃片P时,反射光只有s分量,没有p分量.初始时刻和180$^{\circ}$时,相对A来说仍是p光,没有s分量,这时反射光的光强达到极大值.在A绕z轴转动90$^{\circ}$或者270$^{\circ}$时,P的反射光相对A是p光,没有s分量,而且光以布儒斯特角入射,自然没有反射光. \\
	
\subsubsection{用上述条件测透射光的光强变化}

	随着A的转动,反射光强逐渐增强,到90$^{\circ}$时达到最大,之后继续减弱,在180$^{\circ}$时达到最小值,但是并没有消光,然后继续增强,到270$^{\circ}$时最亮,之后继续减弱,在0$^{\circ}$时回到原点,光强达到最小,并没有消光. \\
	
	根据布儒斯特定律,以布儒斯特角入射的时候,透射光中大部分是s光,很少有p光.在0$^{\circ}$和180$^{\circ}$时,由于入射的光线没有s分量,所以透过的光线很少,而在90$^{\circ}$和270$^{\circ}$时透射光线达到极大值. \\
	
\subsubsection{将A绕轴转到消光位置,然后绕y轴转动,观察反射光的强度变化}

	刚开始的时候消光,但是无论往什么方向转动,光线都会增强. \\
	
	由于偏离了布儒斯特角,反射光强就不为0.这时候光强就会增大.当转到另一个布儒斯特角的时候,还会出现消光现象. \\
	
\subsubsection{在上述条件下观察透射光的强度变化}

	透射光强度变化不太明显.刚开始光点特别暗,后来逐渐增强,转过水平面后又减小. \\
	
	由于反射光总是大于透射光,所以透射光的能量很小,所以变化不明显. \\
	
\subsubsection{检验P的反射光是否是线偏振光}

	偏振片放在平台上转动360$^{\circ}$,看到两次消光.可见P的反射光是线偏振光. \\
	
\subsection{观察双折射现象}

\subsubsection{在平台上放方解石1,转动方解石,观察现象}

	我看到两个光点,在旋转过程中,一个光点不动,另一个光点绕着它转. \\
	
	小孔发出的光线在方解石中被分成了o光和e光.o光符合常规的折射定律,所以不会转动.但是转动过程改变了方解石光轴的位置,所以e光会旋转. \\
	
	o光的视深相对较浅,可见方解石对o光的折射率较大. \\
	
\subsubsection{在平台上放方解石2,转动方解石,观察现象}

	在垂直于光轴的平面上看到一个较模糊的光斑,在其余三个面上有两个光斑.转动方解石,垂直于方解石的面上的光斑不动. \\
	
	由于该光点的光线是沿着光轴入射的,o光和e光不会发生分离,所以看不到两个光斑,也不会看到旋转. \\
	
\subsubsection{利用一透光方向已知的偏振片来判断寻常光和非寻常光的电矢量振动方向}

	用方解石1做实验.在转动偏振片时,o光和e光依次消失.o光和e光消失的角度相差90$^{\circ}$. \\
	
	实验中我并不知道偏振片的透光方向,所以无从判断偏振方向,但是我可以获知o光和e光的振动方向相互垂直. \\

\subsection{观察光线透过$\lambda / 2$片之后的现象}

\subsubsection{在光源和观察者之间放入P,转动O,观察透射光的现象.加入检偏器A,转动A,观察透射光的现象}

	转动P,始终可以看到同样明亮的透射光. \\
	
	加入A之后,转动360$^{\circ}$,在204$^{\circ}$和24$^{\circ}$出现了两次消光现象,在115$^{\circ}$和295$^{\circ}$方向发现光强最大.四个数据之间间隔约90$^{\circ}$. \\
	
	自然光在经过P之后变成了线偏振光,再加入A之后,当A和P正交的时候,人眼就观察不到透射光了. \\
	
\subsubsection{使P的投射方向垂直,转A达到消光.加入$\lambda / 2$片,将其转动360$^{\circ}$,观察消光次数}

	出现了四次消光,分别位于185$^{\circ}$,274$^{\circ}$,4$^{\circ}$,93$^{\circ}$.四个数据间隔约$^{\circ}$. \\
	
	在转动玻片的过程中,会出现四次玻片o轴或e轴和P偏振方向平行的情况,此时玻片透过的光仍为线偏振光,且偏振方向不变.于是就会出现消光现象. \\
	
\subsubsection{转动$\lambda / 2$片,破坏消光,再转动A,观察现象}

	转动A的过程中出现了两次消光,分别为87$^{\circ}$和268$^{\circ}$. \\
	
	线偏振光通过$\lambda / 2$片之后仍为线偏振光,故有两次消光,且相隔约180$^\circ$. \\
	
\subsubsection{使P$\perp$A,转动$\lambda / 2$片,转动使消光.然后把P转动$\theta$,破坏消光.沿着相反方向转动A,直到消光,记录A转过的角度}

	数据见表一,$\theta_P$为P转动之后的角度度数,$\theta$为P的角度差,$\theta_A$为A转动之后的角度度数,$\theta'$为A的角度差,$\Delta \phi$为线偏振光转过的角度. \\
	
\begin{table}
\centering

	\begin{tabular}{|c|c|c|c|c|}
	\hline
	327$^{\circ}$ & 0$^{\circ}$ & 9$^{\circ}$ & 0$^{\circ}$ & 0$^{\circ}$ \\
	\hline
	312$^{\circ}$ & 15$^{\circ}$ & 353$^{\circ}$ & 16$^{\circ}$ & 31$^{\circ}$ \\
	\hline
	297$^{\circ}$ & 30$^{\circ}$ & 339$^{\circ}$ & 30$^{\circ}$ & 60$^{\circ}$ \\
	\hline
	282$^{\circ}$ & 45$^{\circ}$ & 323$^{\circ}$ & 46$^{\circ}$ & 91$^{\circ}$ \\
	\hline
	267$^{\circ}$ & 60$^{\circ}$ & 310$^{\circ}$ & 59$^{\circ}$ & 119$^{\circ}$ \\
	\hline
	252$^{\circ}$ & 75$^{\circ}$ & 293$^{\circ}$ & 76$^{\circ}$ & 151$^{\circ}$ \\
	\hline
	237$^{\circ}$ & 90$^{\circ}$ & 279$^{\circ}$ & 90$^{\circ}$ & 180$^{\circ}$ \\
	\hline
	
	\end{tabular}
	\caption{插入$\lambda / 2$片,转动A,出现消光的角度数据表}
\end{table}
	
	线偏振光经过$\lambda / 2$片之后,出射光仍为线偏振光,但是振动方向有所改变.线偏振光经过$\lambda / 2$片之后会旋转$\theta / 2$角度. \\

\subsection{用$\lambda / 4$片产生椭圆偏振光}

	数据见表二,$\theta_P$为P转动之后的角度度数,$\theta$为P的角度差. \\
	
\begin{table}
\centering

	\begin{tabular}{|c|c|c|c|}
	\hline
	$\theta_P$ & $\theta$ & 现象 & 光的偏振状态 \\
	\hline
	237$^{\circ}$ & 0$^{\circ}$ & 有两次消光 & 线偏振光 \\
	\hline
	222$^{\circ}$ & 15$^{\circ}$ & 有两次强光,两次弱光,现象明显 & 椭圆偏振光 \\
	\hline
	207$^{\circ}$ & 30$^{\circ}$ & 有两次强光,两次弱光,现象不明显 & 椭圆偏振光 \\
	\hline
	192$^{\circ}$ & 45$^{\circ}$ & 光强无变化 & 圆偏振光 \\
	\hline
	177$^{\circ}$ & 60$^{\circ}$ & 有两次强光,两次弱光,现象不明显 & 椭圆偏振光 \\
	\hline
	162$^{\circ}$ & 75$^{\circ}$ & 有两次强光,两次弱光,现象明显 & 椭圆偏振光 \\
	\hline
	147$^{\circ}$ & 90$^{\circ}$ & 有两次消光 & 线偏振光 \\
	\hline
	
	\end{tabular}
	\caption{插入$\lambda / 4$片,转动A,实验现象表}
\end{table}

	当角度为0$^{\circ}$或90$^{\circ}$时,出射光只有e分量或者o分量,仍然为线偏振光.当角度为15$^{\circ}$,30$^{\circ}$,60$^{\circ}$,75$^{\circ}$时,e分量和o分量都不为0,且相位差90$^{\circ}$,所以为椭圆偏振光.当角度为45$^{\circ}$出射光的e分量和o分量振幅相等且相位差90$^{\circ}$,所以为圆偏振光. \\
	
\subsection{产生并检验椭圆偏振光和部分偏振光}

\subsubsection{产生椭圆偏振光和部分偏振光}

	椭圆偏振光在上题中已经产生,光学元件顺序为:钠光灯,起偏器,$\lambda / 4$片. \\
	
	部分偏振光可以用玻璃片堆转过一个角度,自然光斜射来得到. \\
	
\subsubsection{检验椭圆偏振光和部分偏振光}

	首先加入P,旋转P,透射光强度发生变化,但是没有消光.旋转P到光最强的角度,然后加入A.旋转A至消光.再加入$\lambda / 4$片,旋转$\lambda / 4$片直到消光.此时P透过的线偏振光和$\lambda / 4$片的o轴或者e轴重合.撤掉P,旋转A,这个时候如果有消光,则说明光线是椭圆偏振光.否则是部分偏振光. \\
	
\subsection{显色偏振现象}

\subsubsection{转动使P$\perp$A,然后加入并旋转样品,使消光.然后通过A观察不同层度胶带的透射光颜色}

	各条带的颜色依次为:橙色,黄色,绿色,浅绿色,紫色. \\
	
\subsubsection{转动A,直到A//P,观察胶带颜色变化规律}

	转动过程中条带颜色逐渐变暗,到达45$^{\circ}$变成白色.由于有背景白光干扰,可以认为这个时候已经出现了消光.然后继续转动,颜色开始变深,出现了浅蓝,深蓝,紫色,紫红色,黄色.两次的颜色正好互为反色. \\
	
\section{分析与讨论}

	这个实验没有太多定量内容,误差来源主要是实验操作上的失误,仪器精度不足和仪器的工差. \\
	
\begin{itemize}
	\item 在光具座实验中,$\lambda / 2$片和$\lambda / 4$片均没有达到特别好的效果,导致在很多应该完全消光的地方没有完全消光.
	\item 在平台实验中,由于仪器精度所限,角度测量并不准.导致结论也有一定的不确定性.
	\item 外光具座实验并没有提供准直的设备,很难判断光路是否和镜面垂直,导致$\lambda / 2$片和$\lambda / 4$片和光路有一定角度,最后导致很大的测量误差.
	\item 在产生部分偏振光的实验中,由于玻璃的层数不够,产生的部分偏振光效果很差,在转动偏振片的过程中,仅能观察到微弱的光线强弱变化.
\end{itemize}

	以上均是我认为这个实验可以改进的地方. \\
	
\section{感悟和收获}

	虽然老师说这次是"复习课",但其实我这学期并没有选修光学.虽然事先阅读了想关知识,但是还是对这个实验的很多地方不甚了解.做实验之前很多现象其实并不知道,可以说是一边做实验一边学习.但是实验并不困难,我也做出了应该出现的现象.特别是最后的分辨椭圆偏振光和部分偏振光的实验,收获颇多.不仅仅是实验技能上的收获,而且对我来说,在知识上的收获同样不可小觑! \\

\end{CJK*}
\end{document}
