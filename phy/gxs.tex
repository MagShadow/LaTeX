\documentclass[a4paper,11pt]{article}
\usepackage{CJKutf8}
\usepackage{indentfirst}
\usepackage{graphicx}
\usepackage{amsmath}
\usepackage{longtable}
\begin{document}
\begin{CJK*}{UTF8}{gbsn}

\title{\textbf{实验十八\ 弗兰克-赫兹实验}}
\author{1500011424  高学诗\\周五11组7号}
\date{2016年10月28日}
\maketitle

\section{实验数据及处理}

\subsection{Ar管测量结果}
实验条件:$U_{F}=2.64\,$V,$U_{Kg_1}=2.42\,$V,$U_{g_{2}p}=7.50$V。\par
粗测各峰值的位置的结果如Table\:1:\\
\begin{table}[htbp]
\caption{粗测Ar管各峰值的位置}
\centering
\\[4pt]
\begin{tabular}{|c|c|c|c|c|c|c|}
\hline
$n$ & 1 & 2 & 3 & 4 & 5 & 6 \\
\hline
$U_{Kg_2}$/V & 16.9 & 28.3 & 40.5 & 52.8 & 65.7 & 79.7\\
\hline
\end{tabular}
\end{table}
\par
细测$U_{out}$与$U_{Kg_2}$的关系,结果如Table\:2(峰值以*号标出):\\
\begin{longtable}{|c|c|c|c|c|c|c|c|}
\caption{Ar管$U_{out}$与$U_{Kg_2}$的关系}
\centering
%\begin{longtable}{|c|c|c|c|c|c|c|c|}
\hline
$U_{Kg_2}$/V & 0.0 & 6.9 & 8.7 & 9.6 & 10.9 & 11.7 & 12.3 \\
\hline
$U_{out}$/mV & -1.3 & -1.2 & -0.4 & 0.5 & 7.5 & 13.5 & 18.0 \\
\hline\hline
$U_{Kg_2}$/V & 12.9 & 13.3 & 13.7 & 14.2 & 14.6 & 15.0 & 15.4 \\
\hline
$U_{out}$/mV & 21.0 & 22.3 & 23.1 & 24.8 & 25.4 & 26.2 & 26.5 \\
\hline\hline
$U_{Kg_2}$/V & 15.7 & 15.9 & 16.2 & 16.5* & 16.7 & 17.0 & 17.3 \\
\hline
$U_{out}$/mV & 26.7 & 27.0 & 27.2 & 27.3 & 26.6 & 26.5 & 26.2 \\
\hline\hline
$U_{Kg_2}$/V & 17.8 & 18.5 & 19.2 & 19.9 & 20.7 & 21.2 & 21.9 \\
\hline
$U_{out}$/mV & 25.5 & 22.7 & 18.3 & 13.5 & 7.9 & 5.8 & 3.0 \\ 
\hline\hline
$U_{Kg_2}$/V & 22.9 & 23.8 & 24.6 & 25.1 & 25.8 & 26.2 & 26.5 \\
\hline 
$U_{out}$/mV & 3.4 & 10.3 & 23.1 & 31.9 & 42.5 & 47.6 & 49.5 \\
\hline\hline 
$U_{Kg_2}$/V & 26.7 & 27.1 & 27.3 & 27.6 & 27.8 & 28.0 & 28.2* \\
\hline  
$U_{out}$/mV & 51.4 & 53.4 & 54.1 & 55.5 & 56.2 & 56.4 & 56.5 \\
\hline\hline 
$U_{Kg_2}$/V & 28.4 & 28.7 & 29.1 & 29.5 & 30.1 & 30.8 & 31.3 \\
\hline	
$U_{out}$/mV & 56.3 & 55.9 & 54.3 & 50.8 & 44.3 & 32.5 & 23.4 \\
\hline\hline	
$U_{Kg_2}$/V & 32.1 & 32.8 & 33.4 & 33.9 & 34.3 & 35.0 & 35.6 \\
\hline		
$U_{out}$/mV & 11.4 & 3.8 & 0.5 & 0.5 & 2.3 & 15.5 & 30.0 \\
\hline\hline	
$U_{Kg_2}$/V & 36.1 & 36.6 & 37.1 & 37.6 & 38.0 & 38.4 & 38.9 \\
\hline		
$U_{out}$/mV & 43.5 & 52.6 & 64.0 & 70.4 & 75.2 & 81.1 & 84.5 \\
\hline\hline
$U_{Kg_2}$/V & 39.2 & 39.6 & 39.9* & 40.1 & 41.3 & 41.9 & 42.6 \\
\hline		
$U_{out}$/mV & 85.8 & 87.5 & 88.5 & 88.4 & 81.7 & 72.9 & 58.0 \\
\hline\hline	
$U_{Kg_2}$/V & 43.1 & 43.8 & 44.4 & 45.0 & 45.6 & 46.0 & 46.6 \\
\hline		
$U_{out}$/mV & 43.6 & 28.7 & 15.5 & 8.0 & 5.6 & 8.4 & 19.4 \\
\hline\hline	
$U_{Kg_2}$/V & 47.2 & 47.6 & 48.0 & 48.4 & 49.1 & 49.3 & 49.5 \\
\hline		
$U_{out}$/mV & 36.2 & 48.0 & 58.6 & 72.0 & 87.1 & 93.1 & 97.4 \\
\hline\hline	
$U_{Kg_2}$/V & 49.8 & 50.2 & 50.5 & 51.0 & 51.3 & 51.6 & 51.8 \\
\hline		
$U_{out}$/mV & 102.8 & 110.2 & 112.5 & 120.4 & 123.6 & 126.7 & 127.3 \\
\hline\hline	
$U_{Kg_2}$/V & 52.0 & 52.3 & 52.5* & 52.8 & 53.1 & 53.6 & 54.2 \\
\hline		
$U_{out}$/mV & 128.9 & 129.9 & 130.0 & 129.4 & 128.2 & 123.2 & 113.2 \\
\hline\hline	
$U_{Kg_2}$/V & 54.8 & 55.3 & 56.0 & 56.8 & 57.6 & 58.7 & 59.4 \\
\hline		
$U_{out}$/mV & 98.8 & 84.4 & 65.7 & 46.8 & 34.2 & 38.6 & 55.4 \\
\hline\hline	
$U_{Kg_2}$/V & 60.0 & 60.7 & 61.1 & 61.6 & 62.1 & 62.6 & 63.0 \\
\hline		
$U_{out}$/mV & 71.8 & 93.0 & 105.0 & 119.2 & 134.7 & 146.0 & 155.5 \\
\hline\hline	
$U_{Kg_2}$/V & 63.3 & 63.6 & 64.0 & 64.5 & 64.7 & 65.0 & 65.2* \\
\hline		
$U_{out}$/mV & 160.5 & 165.6 & 171.1 & 176.0 & 177.5 & 179.2 & 179.7 \\
\hline\hline	
$U_{Kg_2}$/V & 65.4* & 65.6 & 65.9 & 66.2 & 66.6 & 67.2 & 67.9 \\
\hline		
$U_{out}$/mV & 179.7 & 179.4 & 177.9 & 175.7 & 170.8 & 162.2 & 146.6 \\
\hline\hline	
$U_{Kg_2}$/V & 68.6 & 69.4 & 70.3 & 71.1 & 71.8 & 72.5 & 73.1 \\
\hline		
$U_{out}$/mV & 129.4 & 108.7 & 96.3 & 95.2 & 109.5 & 122.4 & 136.8 \\
\hline\hline	
$U_{Kg_2}$/V & 74.0 & 74.6 & 75.2 & 75.7 & 76.2 & 76.8 & 77.4 \\
\hline		
$U_{out}$/mV & 162.5 & 177.5 & 192.2 & 203.4 & 215.0 & 225.5 & 234.6 \\
\hline\hline	
$U_{Kg_2}$/V & 77.8 & 78.1 & 78.4 & 78.6* & 79.1 & 79.7 & 80.7 \\
\hline		
$U_{out}$/mV & 238.6 & 239.3 & 240.0 & 240.1 & 238.8 & 228.7 & 215.4 \\
\hline\hline	
$U_{Kg_2}$/V & 81.5 & 82.2 &&&&& \\
\hline		
$U_{out}$/mV & 202.2 & 190.1 &&&&& \\
\hline
\end{longtable}
%\end{longtable}
\par
利用MATLAB,用细测的数据作出Ar管的$U_{out} - U_{Kg_2}$图如下页图Figure\:1所示。
\begin{figure}[htbp]
\centering
\caption{Ar管的$U_{out}$与$U_{Kg_2}$关系图}
\includegraphics[scale=.4]{jiao.JPG} 
\end{figure}

\subsection{Hg管测量结果}

\subsection{计算Ar管和Hg管的第一激发电位及其不确定度}

\subsection{改变Hg管的反向电压$U_3$,细测后两个峰}

\section{分析与讨论}

\subsection{散热吸热不是误差主要来源}
从实测数据看,如果实验全过程中散热、吸热没有达到补偿,冰的熔化热结果不一定偏离“合理”的数据范围。\\ \\
我做的另一组实验,水的初温选择的是41$^\circ$C左右,达到稳定后刚刚到室温附近,全过程中散热、吸热并没有达到补偿,系统温度随时间变化的关系如Table3所示。冰的初温为$T_1=-13.0^\circ$C,水和冰的质量分别为$m_0=174.73$g和$m=24.27$g,计算可得$L=3.38\times 10^5$J/kg,未偏离“合理”数据范围。这说明我们的量热器绝热性能很好,吸热散热并不是主要误差来源。

\begin{table}[htbp]
\centering
\begin{tabular}{|c|c|c|c|c|c|c|c|c|c|}
\hline
$t$/s & $T$/$^\circ$C & $t$/s & $T$/$^\circ$C & $t$/s & $T$/$^\circ$C & $t$/s & $T$/$^\circ$C & $t$/s & $T$/$^\circ$C\\
\hline
0 & 41.7 & 205 & 36.2 & 255 & 30.3 & 305 & 27.6 & 355 & 26.6\\
\hline
30 & 41.6 & 210 & 35.0 & 260 & 29.8 & 310 & 27.4 & 360 & 26.6\\
\hline
60 & 41.4 & 215 & 34.6 & 265 & 29.4 & 315 & 27.3 & 365 & 26.6\\
\hline
90 & 41.3 & 220 & 34.1 & 270 & 29.0 & 320 & 27.2 & 370 & 26.5\\
\hline
120 & 41.2 & 225 & 33.3 & 275 & 28.7 & 325 & 27.0 & 380 & 26.5\\
\hline
150 & 41.2 & 230 & 32.8 & 280 & 28.5 & 330 & 26.9 & 390 & 26.5\\
\hline
180 & 41.1 & 235 & 32.3 & 285 & 28.3 & 335 & 26.8 & 400 & 26.6\\
\hline
190 & 39.2 & 240 & 31.9 & 290 & 28.1 & 340 & 26.7 & 410 & 26.6\\
\hline
195 & 38.9 & 245 & 31.3 & 295 & 27.9 & 345 & 26.7 & 420 & 26.6\\
\hline
200 & 37.8 & 250 & 30.9 & 300 & 27.7 & 350 & 26.6 & 430 & 26.6\\
\hline
\end{tabular}\\
\caption{第二次实验系统温度随时间的变化关系}
\end{table}

\subsection{误差主要来源}
\subsubsection{水和冰的质量}
在投入冰的瞬间,可能溅出水,在搅拌的过程中也可能会溅出水,这会对冰熔化的过程本身造成影响,也会使冰质量的测量的相对误差增大。\\ \\
如果从保温桶中取出冰到向水中投入冰间隔过长,冰会先熔化一部分,这对于冰和水的质量也是有影响的。

\subsubsection{冰的初温}
取出冰后,冰从外界吸热,温度会上升,这会带来一定的误差。

\subsubsection{温度计的热容}
我们在计算中忽略的温度计的热容会带来误差。

\subsubsection{冰熔化的时间}
对冰熔化的时间的判断也会影响到最后的结果。

\subsection{粗略修正散热的方法——补偿法}
在本实验中,系统与外界的热传递无法避免,我们可以通过精密的手段测量交换的热量,但是条件不足的情况下只能通过补偿法等方法来减小吸热散热对实验的影响。在其他实验中,也有可能遇到这种情况,一些因素影响我们的实验(比如电学实验中电表的内阻),这时当条件不足、不能测量这个影响时,需要考虑补偿的方法减小这种影响。

\subsection{实测熔化热小于文献值的原因}
从公式上看,各项测量值小于真实值可能导致结果小于文献值。原因可能有以下几点:
\begin{itemize}
\item对冰熔化的时间的判断不准确(或者未搅拌等),冰熔化时系统的温度低于记录的温度;
\item忽略温度计的热容;
\item冰的初温高于保温桶内温度。
\end{itemize}


\section{收获与感想}
我一向认为测定冰的熔化热的实验是非常考验人的一个实验,它要求实验者具备很强的动手能力,需要细致的操作,还需要应用一些方法减小散热等因素对实验的影响。在高中的时候我做过几次这个实验,但是结果不是很理想(传说中数量级正确就行的实验)。这次实验,我仍然在操作上做得不是很好,第一次实验取出冰过了十几秒才投入水中,没等冰熔化后水温上升就停止记录数据等等。我得到的结果与文献值很接近,不是因为实验过程完美,我想应该是误差有正有负抵消掉了,虽然结果很好,但是不确定度应该比其他同学大吧。物理是一门实验科学,实验是必须掌握的能力,我会改正我的错误,提高自己的实验水平。

\end{CJK*}
\end{document}