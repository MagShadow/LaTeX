% Physics experiment report
% 20/Oct/2016

\documentclass[a4paper,10pt,notitlepage]{report}

\usepackage{CJKutf8}
\usepackage{amsmath}
\usepackage{indentfirst}
\usepackage{graphicx}

\setlength{\parindent}{2em} 

\begin{CJK*}{UTF8}{gbsn}
\begin{document}

\title{测量误差和数据处理实验报告}
\author{秦光辉\ 9组3号}
\maketitle

\section*{一、实验数据处理}

	表一是实验数据.D是钢杯外径,d是钢杯内径,H是外部杯高,h是内部杯高. \\
	
\begin{table}[htbp]
\centering
	\begin{tabular}{|c|c|c|c|c|}
	
		\multicolumn{5}{c}{Table 1\ 游标卡尺测钢杯含钢体积} \\
		\hline
		item & D/cm & d/cm & H/cm & h/cm \\
		\hline
		零点读数 &  \\
		\hline
		\# &  \\
		\hline
		1 & \\
		\hline
		2 & \\
		\hline
		3 & \\
		\hline
		6 &  \\
		\hline
		均值 & \\
		\hline
		标准差 & \\
		\hline
		考虑仪器允差后的标准差 & \\
		\hline
		修正零点后的平均值 & \\
		\hline

	\end{tabular}
\end{table}

	测量结果为 \\
	
\begin{align}
	\bar{D} \pm \sigma_{D} &= \\
	\bar{d} \pm \sigma_{d} &= \\
	\bar{H} \pm \sigma_{H} &= \\
	\bar{h} \pm \sigma_{h} &= 
\end{align}

	计算结果是 \\
	
\begin{align}
	\bar{V} = \frac{\pi}{4}(\bar{D}^2 \bar{H} - \bar{d}^2 \bar{h}) = 
\end{align}

	表二是是螺旋测微器测小钢球体积的实验数据.零点为$d_0$. \\
	
\begin{table}[htbp]
\centering
	\begin{tabular}{|c|c|c|c|c|c|c|}
	
		\multicolumn{7}{c}{Table 2\ 螺旋测微器测小钢球体积实验数据} \\
		\multicolumn{7}{l}{\scriptsize{unit: cm}} \\
		\hline
		\# & 1 & 2 & 3 & 4 & 5 & 6 \\
		\hline
		\\
		\hline

	\end{tabular}
\end{table}
	
\begin{equation}
	d_0 = 
\end{equation}
	
	数据处理过程见式(7)到(13),$\sigma_{\bar{d}}$是d平均值的标准差,$\sigma_d$是考虑仪器允差之后d的标准差,$\sigma_V$是V的标准差. \\
		
\begin{align}
	&\bar{d} = \\
	&\sigma_{\bar{d}} = \\
	&\sigma_d = \\
	&\bar{d} \pm \sigma_d = \\
	&V = \frac{\pi \bar{d}^3}{6} \\
	&\sigma_V = \\
	&V \pm \sigma_V = 
\end{align}
	
\section*{二、习题}
\subsection*{1}
	
	(1)0.0001cm 一位 \\
	
	(2)1.000s 五位 \\
	
	(3)2.7 $\times$ $10^{20}$ J 二位 \\
	
	(4)980.120 $\times cm \cdot s^{-2}$ 六位 \\

\subsection*{2}

	(1)

\begin{align}
	&\frac{1}{\frac{1}{a} - \frac{1}{b}} cm = \frac{1}{0.10 \underline{10} - 0.001000 \underline{10}} cm = \frac{1}{0.10\underline{00}} cm = 10.0 cm  \\
	&c \pm e_c = 10.0 \pm 0.1
\end{align}

	(2)

	x的极限误差 \\
	
\begin{equation}
	e_x = 0.01
\end{equation}

	且有

\begin{align}
	&e_x \times \frac{dy}{dx} | _{x = 9.24} = -1.5 \times 10^{-38} \\
	&y = 8 \times 10^{-38} \\
	&y \pm e_y \pm 2 \times 10^{-38}
\end{align}
	
	(3)

\begin{align}
	&e_x = 0.1 \\
	&\frac{dy}{dx} | _{x = 56.7} = 0.0176 \\
	&e_y = e_x * 0.0176 = 0.00176 \\
	&y = 4.038 \\
	&y \pm e_y = 4.038 \pm 0.002
\end{align}

	(4)
	
\begin{align}
	&e_x = 1' \\
	&e_y = \frac{dy}{dx} | _{x = 56.7} \times 1' = -4.75 \times 10^{-5} \\
	&y = 0.98657 \\
	&y \pm e_y = 9.8657 \pm 0.0005 \times 10^{-1}
\end{align}

\subsection*{3}

	(1)

\begin{align}
	&\frac{\partial \rho}{\partial m_1} = - \frac{\rho_0 m_2}{(m_1 - m_2)^2} \\
	&\frac{\partial \rho}{\partial m_2} = \frac{\rho_0 m_1}{(m_1 - m_2)^2} \\
	&\sigma_\rho = \sqrt{(\frac{\partial \rho}{\partial m_1} \times \sigma_{m_1})^2 + (\frac{\partial \rho}{\partial m_2} \times \sigma_{m_2})^2} \\
	&= \frac{\rho_0}{(m_1 - m_2)^2}\sqrt{(m_2 \sigma_{m_1})^2 + (m_1 \sigma_{m_2})^2}
\end{align}

	(2)
	
\begin{align}
	&\frac{\partial y}{\partial a} = \frac{b}{a(a+b)} \\
	&\frac{\partial y}{\partial b} = \frac{a}{b(a+b)} \\
	&\sigma_y = \sqrt{(\frac{\partial y}{\partial a} \times \sigma_a)^2 + (\frac{\partial y}{\partial b} \times \sigma_b)^2} \\
	& = \frac{1}{|a+b|}\sqrt{(\frac{b}{a}\sigma_a)^2 + \frac{a}{b}\sigma_b)^2}
\end{align}

\subsection*{4}



\section*{三、感悟和收获}

	伏安特性曲线是表征材料性质的基本物理量之一,虽然我在实验中只测量了两种常见的元件的曲线,但是我也体会到了这个实验的一些要点所在.在电学实验中,我不能急躁地直接测量并记录数据,要严格按照规范操作,反复检查电路,并随时注意自己的安全.更改电路一定要按照电路图断电操作,这是一切电学实验都要遵守的规范. \\
	
	我在实验中测量了过多的数据,这些数据并不是特别有用.我认识到在记录数据之前,应该先粗测一次,获得曲线的大致形状之后再动手,这样可以避免很多浪费. \\

\end{CJK*}
\end{document}
