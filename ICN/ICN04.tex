\documentclass[a4paper,12pt,notitlepage]{article}

\usepackage{CJKutf8}
\usepackage{indentfirst}

\setlength{\parindent}{2em} 

\begin{CJK*}{UTF8}{gbsn}
\begin{document}

\title{ICN\ 04\ 数据信号及传输}
\author{整理:hiaoxui}
\maketitle

\section{数据,信号和传输}

\begin{itemize}
	\item 分类
	\begin{itemize}
		\item 直接和间接
		\item 信道通信方式
		\begin{itemize}
			\item 单工:单向
			\item 半双工:双向但不同时
			\item 全双工:双向且可以同时
		\end{itemize}
	\end{itemize}
	\item 数据:携带有意义的实体
	\begin{itemize}
		\item 数字信号
		\item 模拟数据
	\end{itemize}
	\item 信号:电子或电磁编码
	\begin{itemize}
		\item 载波:运载数据的实体
		\item 离散信号
		\item 连续信号
		\item 周期信号
		\begin{itemize}
			\item amplitude
			\item periodic/frequency
			\item phase
		\end{itemize}
		\item 信号的强度
		\begin{itemize}
			\item 衰耗和增益
			\item 衡量功率的相对大小计量单位:分贝
			\begin{equation}
				D = 10 \log_{10}\frac{p1}{p2}
			\end{equation}
		\end{itemize}
	\end{itemize}
	\item 传输:通过信号的传播和处理进行的数据通信
	\begin{itemize}
		\item 模拟传输
		\begin{itemize}
			\item 模拟数据->电话->模拟信号
			\item 数字信号->modem->模拟信号
			\item 需要用放大器放大失真信号(噪声也会放大)
		\end{itemize}
		\item 数字传输
		\begin{itemize}
			\item 模拟数据->modem->数字信号
			\item 数字信号->编码解码器->数字信号
			\item 需要用中继器还原真实数据
		\end{itemize}
		\item 传输减损:振幅变小
		\item 传输失真:畸变
		\item 噪声
		\begin{itemize}
			\item 信噪比
			\begin{equation}
				R_{S/N}=10log_{10}\frac{S}{N}
			\end{equation}
		\end{itemize}
	\end{itemize}
\end{itemize}

\section{信道容量}
\begin{itemize}
	\item 指定条件下信道传输数据的能力
	\begin{itemize}
		\item data rate:bps
		\item bandwidth:Hz
	\end{itemize}
	\item 比特率和波特率
	\begin{itemize}
		\item 比特率:每秒的比特数,bps
		\item 符号率/波特率:调制速率,单位时间内传输的波形个数
		\begin{itemize}
			\item 设一个波形持续时间为T,则
			\begin{equation}
				D_{baud} = \frac{1}{T}
			\end{equation}
		\end{itemize}
	\end{itemize}
	\item Nyquist Theory
	\begin{itemize}
		\item 离散无噪声的数字通道信道容量为
		\begin{equation}
			C = 2W\log_2L
		\end{equation}
		\item W为带宽
		\item L为代码的进制数
		\item C单位是bps
	\end{itemize}
	\item Sahnnon Theory
	\begin{itemize}
		\item 在信号平均功率有限的白噪声(通信系统内部产生的噪声)信道中,信道容量为
		\begin{equation}
			C = W\log_2(1+S/N)
		\end{equation}
	\end{itemize}
\end{itemize}

\section{数字编码技术}
\begin{itemize}
	\item 基带传输与通带传输
	\begin{itemize}
		\item 基带信号:基本频带信号,来自信源
		\item 通带信号:基带信号通过载波调制之后,信号频率范围搬移到较高频段以便在信道中传输
	\end{itemize}
	\item 数字数据和数字信号
	\begin{itemize}
		\item 直接传输:可能失真而无法识别
		\item 不归零:在一定时间内信号没有变换(没有返回零电压)
		\begin{itemize}
			\item NRZ-L 用正负电压表示两个二进制位
			\item NRZI 在一位时间内维持一常量电压脉冲
			\begin{itemize}
				\item 简单,带宽利用率高
				\item 缺少同步能力
			\end{itemize}
		\end{itemize}
		\item 双相编码
		\begin{itemize}
			\item Manchester:在每一位中间有一个跳变(0从高到低,1从低到高)
			\item Differential Manchester:数据定义在每一位起始处是否有跳位(0在一开始有变换,1在一开始没变化)
			\item Biphase编码在每一位中间都有跳变,仅用于解决时钟不同步问题,相位变化解决接反问题
		\end{itemize}
	\end{itemize}
	\item 数字数据和模拟信号
	\begin{itemize}
		\item 数字调制:把低频的数字信号变换成选适合于信道传输的处理过程
		\item 数字解调:把信道中已调信号转换成低频数字信号的处理过程
		\item 调制解调:把低频信号转换成高频信号以便于数字信号传播,即将待传信号的频谱进行搬移和还原
		\item 基带传输的调制方式
		\begin{itemize}
			\item AM调幅
			\item FM调频
			\item PM调相
		\end{itemize}
		\item 正交调幅调制:利用正负和相位的不同组合来调制
		\begin{itemize}
			\item QAM-16的特性
			\begin{itemize}
				\item 可供选择的相位有12种,每一种相位有1,2或3种振幅可选
				\item 4bit编码有16中不同组合,16个点中可以对应一种4bit编码
			\end{itemize}
		\end{itemize}
	\end{itemize}
	\item 模拟数据和数字信号
	\begin{itemize}
		\item 数字化:将模拟数据转换成数字信号的过程
		\item 编码解码技术(CODEC)
		\item 脉冲编码调制
		\begin{itemize}
			\item 采样定理:如果以两倍最高有效频率采样的话,可以采集到原信号所有信息
			\item 模拟信号->采样->量化->编码->数字传输->解码->滤波->输出信号
			\item PAM脉冲:时间离散,幅值连续
			\item 量化:把样值信号无限多的可能取值,近似用有限个数的数值表示
			\item 量化级:把样值信号幅度分为很多度量单位,一个度量单位称为一个量化级.
			\item 量化误差:量化值和原样值的幅度差别
			\item PCM输出:时间离散,幅值离散
		\end{itemize}
		\item 增量调制
		\begin{itemize}
			\item 用差值编码,要求相邻的样值之间变化不大(一个量化级)
		\end{itemize}
		\item 差分调制
		\begin{itemize}
			\item 数字化后输出的不是数字化幅度本身,而是和前一个值的差值
		\end{itemize}
	\end{itemize}
	\item 模拟数据和模拟信号
	\begin{itemize}
		\item 模拟数据->modem->模拟信号
		\item 模拟调制:把一个输入信号和频率为$f_c$的载波结合,以便产生一个以$f_c$为中心的信号
		\item 无线传输无法传递基带信号,而且有了modem技术,可以多路复用
	\end{itemize}
\end{itemize}

\end{CJK*}
\end{document}